\usepackage{subfiles}
\usepackage{amsmath}
\usepackage{amsfonts}       % \mathbb{}
\usepackage{amssymb}        % \therefore
\usepackage{fancyhdr}       % \pagestyle{fancy}

\usepackage{tikz}   % tikz
% \usepackage{pgfplots}

\usepackage{hyperref}     % \hypersetup

% 目次を subsubsection まで書く
\setcounter{tocdepth}{3}

% header, footer
\pagestyle{fancy}
\fancyhead[OL]{}            % Odd Left
\fancyhead[OR]{ \leftmark } % Odd Right

\renewcommand{\sectionmark}[1]{\markboth{#1}{#1}}

% enumerate の番号の振り方を変える
\renewcommand{\labelenumi}{[\arabic{enumi}]}
\renewcommand{\labelenumii}{[\arabic{enumi}-\arabic{enumii}]}



% ハイパーリンク設定
\hypersetup{
  % しおりの設定
  % bookmarks = true,			% しおりを作る
  bookmarksnumbered = true,	% しおりのラベルを作る
  bookmarksopen = true,		% しおりのツリーを開く
  bookmarksopenlevel = 4,    	% しおりの深さ
  bookmarkstype = toc,
  % リンクカラーの設定
  colorlinks = false,				% カラーリンク
  anchorcolor = brack,			% アンカーテキストの色指定(デフォルトはblack)
  linkcolor = red,				% 内部参照リンク用のカラー
  linkbordercolor = {1 0 0}, 		% RGBリンクを囲むボックスの色(デフォルトは1 0 0)
  citecolor = green,				% 文献参照用リンク用のカラー 
  filecolor = magenta,			% ローカルファイルリンクの色指定(デフォルトはmagenta)
  urlcolor = magenta,			% 外部参照しているurlの色(デフォルトはmagenta)
  % PDFプロファイルの設定
  pdfborder={1 0 0},			% RGBリンクを囲むボックスの色(デフォルトは1 0 0)
  pdftitle={},					% pdfのタイトル
  pdfauthor={},					% pdfの著者名
  pdfsubject={},				% pdfのサブタイトル
  pdfkeywords = {},				% pdfのキーワード
  % pdfpagemode = UseThumbs	% サムネイル表示
}
